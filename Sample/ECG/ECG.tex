% Document
\documentclass{beamer}
\usetheme{Madrid}
\usepackage{graphicx}
\usepackage{multirow}

% Preamble

\title{ ANALYSING AND FORECASTING ELECTRICITY CONSUMPTION OF GHANA TIME SERIES}
\author{ECF GROUP}
\logo{\includegraphics[height=0.8cm, width=0.8cm]{images/logo.png}}
\institute{Kwame Nkrumah University of Science and Technology}

\date{March 2, 2024}


% Body
\begin{document}
	% Define variables here
	\newcommand{\highlight}{blue}
	
	% Title page
	\begin{frame}
		\titlepage
	\end{frame}
	
	% Table of Content
	\begin{frame}
		\tableofcontents
	\end{frame}
	
	% Names
	\section{Names}
	\begin{frame}
		\begin{table}
			\begin{tabular}{lc}
				\multicolumn{2}{c}{\textbf{MEMBERS}}  \vspace{10pt}    \\
				\textbf{NAME}     & \textbf{INDEX NUMBER} \vspace{10pt}  \\
				DARKO ROCKSON     & 4347020               \\
				APAAH PRINCE      & 4341520               \\
				AMPONG DORCAS     & 4340320               \\
				BOATENG ELIZABETH & 4346020              
			\end{tabular}
		\end{table}
	\end{frame}
	
	
	% Introduction
	\section{Introduction}
	\begin{frame}{INTRODUCTION}
		Ghana has three primarily distribution utilities, two of which are states owned \textcolor{\highlight}{(ECG \& NEDCo)} and one of which is run privately \textcolor{\highlight}{(EPC)}. \vspace{10pt} \pause
		
		Energy mix has primarily consisted of hydro and thermal sources.In 2021 hydro accounting for around \textcolor{\highlight}{34.1\%} of the total power, with the thermal accounting for \textcolor{\highlight}{65.3\%} and renewable accounting for 0.55\% \vspace{10pt} \pause
		
		Total electricity generation almost doubled from 14,068 GWh in 2011 to 22,051 GWh in 2021, representing an annual average growth rate of 11\%. \vspace{10pt} \pause
		
		Total electricity consumption increased from 13,036 GWh in 2017 to 18,067 GWh in 2021 representing an annual average growth rate of 8\% (according to energy commission of Ghana).
	\end{frame}
	
	
	% Description
	\section{Description}
	\begin{frame}{Description}
		\begin{table}
		\begin{tabular}{|l|l|l|l|l|l|l|}
\hline
Min & 1st Quatile & Median & Mean & 3rd Quatile & Max & Var \\ \hline
86.27 & 281.04 & 328.29 & 322.29 & 372.99 & 523.25 & 6647.897 \\ \hline
		\end{tabular}
		\end{table}
		
		We ploted the electricity consumption(KWh per capita) data against time of their collection,Thus from 1971 to 2022 as shown in figure 4.1.
		
		
		
	\end{frame}

	\begin{frame}
		\begin{center}
			\includegraphics[width=0.9\linewidth]{images/image1}
		\end{center}
	\end{frame}
	
	% Stationarity
	
\end{document}