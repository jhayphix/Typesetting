% Document class
\documentclass{beamer}
\usecolortheme{default}
\usetheme{Madrid}
\usepackage{graphicx}
\usepackage{subfigure}

% Preamble
\title{Multivariate Time Series Modelling Of Ex-Pump Prices Of Petroleum Products In Ghana}
\subtitle{Chapter 4: Results and Discussions}
\author{Group 41}
\institute{Kwame Nkrumah University of Science and Technology}
\subject{Final Year Project}
\logo{\includegraphics[width=20pt, height=20pt]{images/logo/logoKnust.png}}
\date{\today}

% Body
\begin{document}
	% Define variables
	\newcommand{\startDate}{ January, 2007 }
	\newcommand{\finishDate}{ June, 2015 }
	\newcommand{\numOfObservations}{ 204 }
	
	
	% Insert title here
	\begin{frame}
		\titlepage
	\end{frame}
	
	% Introduction
	\section{Introduction}
	\label{sec:introduction}
	\begin{frame}{Introduction}
		
		\begin{exampleblock}{Objective}
			The purpose of the study is to obtain a suitable model for the ex-pump prices of petroleum products in Ghana. 
		\end{exampleblock} \vspace{5pt}
		
		 To examines how changes in the prices of one product cause changes in the price of others in both the short and long terms. \par \vspace{5pt}
		
		Data spanning \startDate to \finishDate are obtained from the National Petroleum Authority of Ghana, covering four petroleum products; \textcolor{blue}{Gasoline, Gasoil, Kerosene, and Liquefied Petroleum Gas (LPG) }. 
	\end{frame}

	\begin{frame}{Chapter 4: Result And Discussion}
		This chapter analyses and discusses the results. It presents results of the association between the prices of the products considered, namely; \vspace{5pt}
		
		\begin{itemize}
			\item Gasoil
			\item Gasolin
			\item Kerosene 
			\item Liquefied Petroleum Gas (LPG)
		\end{itemize} \vspace{5pt}
		
		All associated tests and models are generated with R 
	\end{frame}
	
	
	% Descriptive Statistics
	\section{Descriptive Statistics}
	\label{sec:Descriptive}
	\begin{frame}{Descriptive Statistics}
		In all, \numOfObservations observations are used (January 2007 to June 2015). 
		Training data of 144 observations (January 2007 to December 2012) for modelling and 
		60 data points (January 2013 to June 2015) for model validations. 
		The 144 observations and 60 data points are arrived at because the data is biweekly in nature. 
		The descriptive statistics of the products are shown in Table 4.1.
	\end{frame}
	
	
	% Trend and Stationarity Test
	
	
	% Differencing
	
	
	% Estimation of VAR or VEC Models
	
	
	% LLS Criteria and Cointergration
	
	% Long and Short Term Equilibrium
	
	
	% Estimation of VEC Model
	
	
	% Model Validation
	
	
	% Forecast of Ex-Pump Prices of Products
	
	
	% Granger Gausality Test
	
	
	% Impurse Response Functions ( IRFs )
	
	% Summary of Results
	
	% Summary of Chapter
	
	
	
\end{document}