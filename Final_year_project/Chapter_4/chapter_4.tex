% Document class
\documentclass{beamer}
\usecolortheme{default}
\usetheme{Madrid}
\usepackage{graphicx}
\usepackage{subfigure}
\usepackage{comment}
\usepackage{enumitem}
\usepackage{amsmath}

% Preamble
\title{Multivariate Time Series Modelling Of Ex-Pump Prices Of Petroleum Products In Ghana}
\subtitle{Chapter 4: Results and Discussions}
\author{Group 41}
\institute{Kwame Nkrumah University of Science and Technology}
\subject{Final Year Project}
\logo{\includegraphics[width=20pt, height=20pt]{images/logo/logoKnust.png}}
\date{\today}

% Customizing table of contents
\setbeamertemplate{section in toc}[sections numbered]
\setbeamertemplate{subsection in toc}[subsections numbered]
\setbeamercolor{section in toc}{fg=blue}
\setbeamercolor{subsection in toc}{fg=blue!70}

% Define variables
\newcommand{\startDate}{January, 2007 }
\newcommand{\finishDate}{June, 2015 }
\newcommand{\numOfObservations}{204 }

\newcommand{\vspaceFive}{\vspace{5pt}}
\newcommand{\vspaceTen}{\vspace{10pt}}
\newcommand{\hspaceFive}{\hspace{5pt}}
\newcommand{\colorHighlight}{blue}
\newcommand{\warning}{red}
\newcommand{\textHighlight}[1]{\textcolor{blue}{#1}}

\newcommand{\VAR}{Vector Autocorelation }
\newcommand{\VARHighlight}{\textHighlight{Vector Autocorelation }}
\newcommand{\VEC}{Vector Error Correlation }
\newcommand{\VECHighlight}{\textHighlight{Vector Error Correlation }}

\newcommand{\mc}[3]{\multicolumn{#1}{#2}{#3}}
\newcommand{\textSubMath}[2]{$(\text{#1})_{\text{#2}}$}
\newcommand{\yLag}[1]{\text{y}_{\text{t}-#1}}
\newcommand{\subTT}[2]{\text{#1}_{\text{#2}}}
\newcommand{\subNT}[2]{#1_{\text{#2}}}
\newcommand{\subTN}[2]{\text{#1}_{#2}}




% Body
\begin{document}
	
	% Insert title here
	\begin{frame}
		\titlepage
	\end{frame}

	% Insert table of content
	\begin{frame}{Table of Contents}
		\tableofcontents
		
	\end{frame}
	
	% --------------------------------------------------------------------
	% Introduction
	% --------------------------------------------------------------------
	\section{Introduction}
	\begin{frame}{Introduction}
		
		\begin{block}{Objective}
			The purpose of the study is to obtain a suitable model for the ex-pump prices of petroleum products in Ghana. 
		\end{block} \vspace{5pt}
		
		 To examines how changes in the prices of one product cause changes in the price of others in both the short and long run. \par \vspace{5pt}
		
		Data spanning \startDate to \finishDate are obtained from the National Petroleum Authority of Ghana, covering four petroleum products; \textHighlight{Gasoline, Gasoil, Kerosene, and Liquefied Petroleum Gas (LPG) }. 
	\end{frame}
	
	% --------------------------------------------------------------------
	\begin{frame}{Chapter 4: Result And Discussion}
		This chapter analyses and discusses the results. It presents results of the association between the prices of the products considered, namely; \vspace{5pt}
		
		\begin{itemize}
			\item Gasoil
			\item Gasolin
			\item Kerosene 
			\item Liquefied Petroleum Gas (LPG)
		\end{itemize} \vspace{5pt}
		
		All associated tests and models are generated with R 
	\end{frame}

	% --------------------------------------------------------------------
	% Overview
	% --------------------------------------------------------------------
	\section{Overview}
	\begin{frame}{RoadMap}
		\begin{block}{Start Up}
			\vspaceFive
			\begin{itemize}[label=$\diamond$, leftmargin=2em, itemsep=1em]
				\item Plotting and Descriptive Statistics
				\item Stationarity Test
				\item Differencing If Not Stationary
				\item Plotting of ACF and PACF
			\end{itemize}
			\vspaceFive
		\end{block}
	\end{frame}

	% --------------------------------------------------------------------
	\begin{frame}{RoadMap}
		\begin{block}{Estimation Of Model}
			\vspaceFive
			\begin{itemize}[label=$\diamond$, leftmargin=2em, itemsep=1em]
				\item Lag Length Selection (LLS)
				\item Cointegration Test
				\item Long Run Equilibrium
				\item Short Run Equilibrium
				\item Estimation of VEC Model (If There is cointegration)
				\item Model Validation
				\item Forecast of Ex-Pump prices of Products
			\end{itemize}
			\vspaceFive
		\end{block}
		
	\end{frame}

	
	% --------------------------------------------------------------------
	% Descriptive Statistics
	% --------------------------------------------------------------------
	\section{Descriptive Statistics}
	\begin{frame}{Descriptive Statistics}
		\begin{block}{}
			\vspace{4pt}
			In all, \numOfObservations observations are used (\startDate to \finishDate). \vspace{4pt}
		\end{block} \vspace{5pt}
		
		\begin{block}{}
		Training data of 144 observations (January 2007 to December 2012) for modelling \\ \vspace{5pt}
		
		Testing data of 60 data points (January 2013 to June 2015) for model validations.
		\end{block} \vspace{5pt}
	
		\begin{block}{}
			The descriptive statistics of the products are shown in Table \ref{table:description} on page \pageref{table:description}
		\end{block}
	\end{frame}
	
	% --------------------------------------------------------------------
	\begin{frame}{Summary Statistics}
		\begin{table}[]
			\caption{ \ref{table:description} Summary Statistics}
			\begin{tabular}{lllll}
\hline
Statistics             & GASOIL  & GASOLINE & KEROSENE & LPG     \\ \hline
Mean                   & 122.445 & 123.570  & 82.989   & 94.766  \\
Maximum                & 175.480 & 177.090  & 120.420  & 136.190 \\
Minimum                & 11.600  & 49.170   & 6.470    & 58.500  \\
Standard Deviation     & 32.306  & 31.817   & 27.186   & 20.609  \\ 
Skewness               & -0.201  & 0.1307   & -1.988   & 0.413   \\
Kurtosis               & 3.374   & 2.123    & 6.293    & 2.292   \\
Number of Observations & 144     & 144      & 144      & 144     \\ \hline
			\end{tabular}
			\label{table:description}
		\end{table}
	\end{frame}

	% -------------------------------------------------------------------
	\begin{frame}{Plot of Original Series}
		\begin{figure}
			\subfigure{\includegraphics[height=0.3\textheight, width=0.4\textwidth]{images/plots/plot_original/plot_original_gasoline}}
			\subfigure{\includegraphics[height=0.3\textheight, width=0.4\textwidth]{images/plots/plot_original/plot_original_gasoil}}
			\subfigure{\includegraphics[height=0.3\textheight, width=0.4\textwidth]{images/plots/plot_original/plot_original_kerosene}}
			\subfigure{\includegraphics[height=0.3\textheight, width=0.4\textwidth]{images/plots/plot_original/plot_original_lpg}}
			
			
			\caption{ \ref{plot:original_series} Time Series Plot of the Original Series}
			\label{plot:original_series}
		\end{figure}
		
	\end{frame}
	
	% --------------------------------------------------------------------
	% Trend and Stationarity Test
	% --------------------------------------------------------------------
	\section{Stationarity Test}
	
	\begin{comment}
	\begin{frame}{Trend Test}
		For trend test, We have chosen to apply
		
		\begin{block}{ Mann-Kendall Test}
			H0 : There is no monotonic trend in the dataset over time. \\
			H1 : There is a monotonic trend in the dataset over time. \vspace{5pt}
		\end{block}
		
		\begin{exampleblock}{Sen's Slope Test}
			H0 : There is no tonic trend in the dataset over time. \\
			H1 : There is a tonic trend in the dataset over time. \vspace{5pt}
		\end{exampleblock}
	\end{frame}
	\end{comment}

	\begin{frame}{Staionarity Test}
		We have numerous ways of testing for the presence of a unit root. 
		We have chosen to apply
		
		\begin{block}{Augmented Dickey-Fuller Test}
			H0 : The series is not stationary \\
			H1 : The series is stationary. \vspace{5pt}
		\end{block}
	
		\begin{exampleblock}{Phillips-Perron Unit Root Test}
			H0 : The series is not stationary \\
			H1 : The series is statrionary. \vspace{5pt}
		\end{exampleblock}
		
		\begin{alertblock}{KPSS Test for Level Stationarity}
			H0 : The series is stationary \\
			H1 : The series is not statrionary. \vspace{5pt}
		\end{alertblock}
	\end{frame}
	
	% --------------------------------------------------------------------
	\begin{frame}{Stationarity of Original Series}
		\begin{table}[]
			\caption{ \ref{table:stationarity_original} Univariate URTs of the Original Series}
			\label{table:stationarity_original}
			\begin{tabular}{llllll}
\hline
&  & \multicolumn{2}{l}{(Test Statistics)} & \multicolumn{2}{l}{(P-Values)} \\ \hline

SERIES   & LAG ORDER & ADF    & KPSS      & ADF    & KPSS           \\

GASOLINE & 5         & -2.738 & 2.370     & 0.269  & 0.010          \\

GASOIL   & 5         & -2.450 & 2.437     & 0.389  & 0.010          \\

KEROSENE & 5         & -3.106 & 0.709     & 0.166  & 0.010          \\

LPG      & 5         & -1.975 & 1.497     & 0.587  & 0.010          \\ \hline
			\end{tabular}
		\end{table}
	
	\begin{alertblock}{Is Stationary ?}
		\vspace{5pt}
		It is observed that for ADF, all the p-values of the series are greater than 0.05 and this indicates non stationarity. The KPSS test also showed the same results. We now  difference the series since the series are not stationary.
	\end{alertblock}
	\end{frame}
	
	% --------------------------------------------------------------------
	% Differencing
	% --------------------------------------------------------------------
	\section{Differencing}
	\begin{frame}{Differencing}

			\begin{alertblock}{First Difference}
				\vspace{5pt}
				Since all the series (Gasoline, Gasoil, Kerosene, LPG) are not stationary we perform 1st differencing in order to achieve stationarity;
				The figure \ref{plot:difference_series} below is a plot after the first differencing .
				\vspace{5pt}
			\end{alertblock}
		
	\end{frame}
	
	% --------------------------------------------------------------------
	\begin{frame}{Plot of First Differenced Series}
		\begin{figure}
			\subfigure{\includegraphics[height=0.3\textheight, width=0.4\textwidth]{images/plots/plot_diff/plot_diff_gasoline}}
			\subfigure{\includegraphics[height=0.3\textheight, width=0.4\textwidth]{images/plots/plot_diff/plot_diff_gasoil}}
			\subfigure{\includegraphics[height=0.3\textheight, width=0.4\textwidth]{images/plots/plot_diff/plot_diff_kerosene}}
			\subfigure{\includegraphics[height=0.3\textheight, width=0.4\textwidth]{images/plots/plot_diff/plot_diff_lpg}}
			
			
			\caption{ \ref{plot:difference_series} Time Series Plot of the Original Series}
			\label{plot:difference_series}
		\end{figure}
		
	\end{frame}

	% --------------------------------------------------------------------
	\begin{frame}{Stationarity of First Differenced Series}
		\begin{table}[]
			\caption{ \ref{table:stationarity_first_diff} Univariate URTs of the Differenced Series}
			\label{table:stationarity_first_diff}
			\begin{tabular}{llllll}
	\hline
	&  & \multicolumn{2}{l}{(Test Statistics)} & \multicolumn{2}{l}{(P-Values)} \\ \hline
	SERIES   &LAG ORDER & ADF    & KPSS      & ADF    & KPSS           \\
	
	GASOLINE &5         & -7.781 & 0.031     & 0.010  & 0.10          \\
	
	GASOIL   &5         & -5.537 & 0.045     & 0.010  & 0.10          \\
	
	KEROSENE &5         & -4.493 & 0.263     & 0.010  & 0.10          \\
	
	LPG      &5         & -4.473 & 0.063     & 0.010  & 0.10          \\ \hline
			\end{tabular}
		\end{table}
	
		\begin{exampleblock}{Is Stationary ?}
			\vspace{5pt}
			It is observed that for ADF, all the p-values of the series are less than 0.05 and this
			indicates the stationarity. The KPSS test also showed the same results. We now estimate the models since the series have attained stationarity.
		\end{exampleblock}
	\end{frame}

	% --------------------------------------------------------------------
	\begin{frame}{ACF Plot of First Differenced Series}
		\begin{figure}
			\subfigure{\includegraphics[height=0.3\textheight, width=0.4\textwidth]{images/acf_pacf/acf_pacf_diff/acf_diff_gasoline}}
			\subfigure{\includegraphics[height=0.3\textheight, width=0.4\textwidth]{images/acf_pacf/acf_pacf_diff/acf_diff_gasoil}}
			\subfigure{\includegraphics[height=0.3\textheight, width=0.4\textwidth]{images/acf_pacf/acf_pacf_diff/acf_diff_kerosene}}
			\subfigure{\includegraphics[height=0.3\textheight, width=0.4\textwidth]{images/acf_pacf/acf_pacf_diff/acf_diff_lpg}}
			
			
			\caption{ \ref{plot:acf__diff_series} ACF of the Differenced Series}
			\label{plot:acf__diff_series}
		\end{figure}
		
	\end{frame}
	
	% --------------------------------------------------------------------
	\begin{frame}{PACF Plot of First Differenced Series}
		\begin{figure}
			\subfigure{\includegraphics[height=0.3\textheight, width=0.4\textwidth]{images/acf_pacf/acf_pacf_diff/pacf_diff_gasoline}}
			\subfigure{\includegraphics[height=0.3\textheight, width=0.4\textwidth]{images/acf_pacf/acf_pacf_diff/pacf_diff_gasoil}}
			\subfigure{\includegraphics[height=0.3\textheight, width=0.4\textwidth]{images/acf_pacf/acf_pacf_diff/pacf_diff_kerosene}}
			\subfigure{\includegraphics[height=0.3\textheight, width=0.4\textwidth]{images/acf_pacf/acf_pacf_diff/pacf_diff_lpg}}
			
			
			\caption{ \ref{plot:pacf__diff_series} PACF of the Differenced Series}
			\label{plot:pacf__diff_series}
		\end{figure}
		
	\end{frame}	
	
	
	% --------------------------------------------------------------------
	% Estimation of VAR or VEC Models
	% --------------------------------------------------------------------
	\section{Estimation of VAR/ VEC Models}
	\begin{frame}{What Next After Series is Stationary ?}
		
		\begin{block}{Estimation of VAR/ VEC Models}
			\vspaceFive
			\begin{itemize}[label=$\diamond$, leftmargin=2em, itemsep=1em]
				\item Lag Length Selection (LLS)
				\item Cointegration Test
				\item Long Run Equilibrium
				\item Short Run Equilibrium
				\item Estimation of VEC Model (If There is cointegration)
				\item Model Validation
				\item Forecast of Ex-Pump prices of Products
			\end{itemize}
			\vspaceFive
		\end{block}
		
	\end{frame}
	
	
	
	% --------------------------------------------------------------------
	\begin{frame}{Estimation of VAR/ VEC Models}
		\begin{block}{}
			Estimating parameters of \VARHighlight (VAR) or \VECHighlight (VEC) models require that variables are 
			covariance stationary \vspaceFive
		\end{block} \vspaceFive
		
		\begin{block}{}
			VAR for instance cannot be used if the variables are not stationary. \\
			Also, if the data is non-stationary, the forecast cannot be done because \textHighlight{VAR assumes stationarity}
		\end{block}
		
		\begin{block}{}
			We then test for the long run relationship using \textHighlight{Johansen’s cointegration test}.  \\
			That is if the result confirms that there is a long-run relationship among the variables, 
			we can proceed to the VEC model. 
		\end{block}
		
		\begin{exampleblock}{}
			The first step involved in estimating is to first determine the lag Length or order. 
		\end{exampleblock}
	\end{frame}
	
	
	% --------------------------------------------------------------------
	% LLS Criteria
	% --------------------------------------------------------------------
	\section{LLS Criteria and Cointergration}
	\begin{frame}{Lag Length Selection (LLS) Criteria}
		\begin{block}{}
			LLS is significant for VAR/VEC models since selecting too few intervals to result in a cointegrated error and selecting too many intervals may lead to unnecessary loss of degrees of freedom
		\end{block}
	
		\begin{exampleblock}{Three of the LLS criteria are used, namely ;}
			\begin{itemize}
				\item FPE (Final Prediction Error)
				\item AIC (Akaike Information Criterion)
				\item BIC (Bayesian Information Criterion), aka SC (Schwarz Criterion)
			\end{itemize} 
		\end{exampleblock}
		
		
		\begin{block}{}
			FPE, AIC, and BIC support the inclusion of lag 1 as italicized, and starred in Table \ref{table:lls}. 
		\end{block}
	\end{frame}

	% --------------------------------------------------------------------
	\begin{frame}

		\begin{table}[]
			% Caption and Label
			\caption{ \ref{table:lls} Lag Length Selection Criteria}
			\label{table:lls} \vspaceFive
			
			% Table Lag Length Selection
			\begin{tabular}{llll}
			\hline
			\mc{1}{c}{Lag} & \mc{1}{c}{FPE} & \mc{1}{c}{AIC} & \mc{1}{c}{BIC} \\ \hline
			
			0              & 1.03 x $10^{9}$     & 32.107         & 32.192         \\ \hline
			
			1*             & 117944.1*      & 23.029*        & 23.454*        \\ \hline
			
			2              & 127300.7       & 23.105         & 23.869         \\ \hline
			
			3              & 142926.7       & 23.219         & 24.3224        \\ \hline
			
			4              & 149122.0       & 23.259         & 24.701          \\ \hline
			
			5              & 169942.3       & 23.385         & 25.167          \\ \hline
			
			6              & 156708.8       & 23.297         & 25.419          \\ \hline
			\end{tabular}
		\end{table}
	
		% Description of table
		\begin{block}{}
			From Table \ref{table:lls}, we can rely on information criteria as only one of these three tests; FPE, AIC, and BIC obtained minimum values at the indicated lag.  \\
			The test displays \textHighlight{lag 1} as the optimum. Thus, the lag length for the estimation is 1. 
		\end{block}
	\end{frame}

	% --------------------------------------------------------------------
	\begin{frame}{What Next After Lag Length Selection}
		\begin{block}{}
			\vspaceTen
			Once the unit roots and lag length selections are determined for a time series data, the next step is to inspect whether there exists a \\
			\textHighlight{Cointegration (Long run relationship)} among the variables or not.
			\vspaceTen
		\end{block}
	\end{frame}

	% --------------------------------------------------------------------
	% Cointergration
	% --------------------------------------------------------------------
	\section{Cointegration}
	\begin{frame}{Cointegration : Long Run Relationship}
		Cointegration analysis is important because, if two or more non-stationary variables are cointegrated, a VAR model in the first difference is mis-specified due to the effects of a common trend. The cointegration test determines the type of the regression model to be applied i.e. VAR or VEC
		
		\begin{block}{ Cointegration Test}
			H0 : There is no cointegration equation. \\
			H1 : There is a cointegration equation 
		\end{block}
		
	\end{frame}
	
	% --------------------------------------------------------------------
	\begin{frame}
		
		% Table
		\begin{table}[]
			\caption{ \ref{table:cointegration} Determining the Number of Cointegrated Equations}
			\label{table:cointegration}
			\begin{tabular}{llllll}
				\hline
				\mc{1}{c}{Number} & \mc{1}{c}{} & \mc{1}{c}{Trace} & \mc{1}{c}{}    & \mc{1}{c}{Max-Eigen}  & \mc{1}{c}{}        \\
				
				\mc{1}{c}{of CE} & \mc{1}{c}{Eigenvalues} & \mc{1}{c}{Statictics} & \mc{1}{c}{P-Value} & \mc{1}{c}{Staticstics} & \mc{1}{c}{P-Value} \\ \hline
				
				None*     & 0.358   & 79.102  & 0.000  & 62.959 & 0.000 \\
				
				At most 1 & 0.070   & 16.143  & 0.702  & 10.258 & 0.720 \\
				
				At most 2 & 0.033   & 5.885   & 0.709  & 4.778  & 0.769 \\
				
				At most 3  & 0.008  & 1.107   & 0.293  & 1.107  & 0.293 \\ \hline
			\end{tabular}
		\end{table}
	
		% Description
		\begin{exampleblock}{Conclusion}
			Remarkably, the Trace test and max-Eigen statistics suggest the existence of a cointegrated equation (CE). \\ 
			We shall take into account this fact at the next step. \\
			Since all the series are I(1) and cointegrated, the products ought to be modelled as a VEC model 
		\end{exampleblock}
	\end{frame}

	% --------------------------------------------------------------------
	\begin{frame}
		\begin{block}{}
			As a result, a cointegration relationship is obtained. \\ 
			This throws more light on the long run relationships among the products.\\
			 Consequently, the products; \textHighlight{GASOLINE, GASOIL, KEROSENE, and LPG} prices are linked by a 
			long run equation. 
		\end{block}
	
		\begin{block}{}
			Once the unit roots and lag length selections are determined for a time series data, the next step is to inspect whether there exists a long-run equilibrium relationship among the variables or not.
		\end{block}
	\end{frame}
		
	% --------------------------------------------------------------------
	% Long and Short Run Equilibrium
	% --------------------------------------------------------------------
	
	\section{Long And Short Run Equilibrium}
	\begin{frame}{Long Run Relationship}
		\begin{block}{The cointegrating (long-run) relationship is estimated to be;}
			\vspaceFive
			\textHighlight{GASOLINE} $ = $ $-$ 0.0221 \textHighlight{GASOIL} $ + $  0.027 \textHighlight{KEROSENE} $-$ 0.580 \textHighlight{LPG}
			\vspaceFive 
		\end{block} 
		\vspaceFive
		
		Thus, with GASOLINE price as the endogenous variable, the long-run relationship indicates that the ex-pump prices of the other products have long run effects. \\ \vspaceTen
		
		Specifically, the results indicate that the other products have a negative relation with GASOLINE price in the long run (except KEROSENE), all things being equal.
	\end{frame}

	% --------------------------------------------------------------------
	\begin{frame}{Log Run Equilibrium}
		\begin{block}{}
			The coefficients of the error correction terms (ECT) [Table \ref{table:gasoline_model}, \ref{table:gasoil_model}, \ref{table:kerosene_model}, \ref{table:LPG_model}] show the speed of adjustments of disequilibrium in the period under study. The negative sign associated with the error term is simply a departure in one direction. These are satisfying as they imply convergence in the long run. That is, deviation from the long run is corrected
		\end{block}.
	\end{frame}

	% -------------------------------------------------------------------
	\begin{frame}
		% Table

		\begin{table}[]		
			% Gasoline table
			\caption{ \ref{table:gasoline_model} Gasoline Model}
			\label{table:gasoline_model}
			\begin{tabular}{llll}
				\hline
				Parameters      & Coefficient & S.E   & t-satistics \\ \hline
				
				\mc{4}{l}{\textHighlight{Gasoline Model}}                  \\ 
				
				\textSubMath{Gasoline}{t-1} & 0.691   & 0.189 & 3.650*  \\ 
				\textSubMath{Gasoil}{t-1}   & -0.0221  & 0.386 & -1.561*  \\ 
				\textSubMath{Kerosene}{t-1} & 0.027   & 0.091 & 0.294   \\ 
				\textSubMath{LPG}{t-1}     & -0.580   & 0.262 & -2.211   \\ 
				Constant        & 0.006       & 0.805 & 0.007    \\ 
				ECT             & -0.613      & 0.145 & -11.118*  \\ 
				\hline
			\end{tabular}
		\end{table}
	
		% Description
		\begin{block}{}
			The negative sign associated with the coefficients of the error term of GASOLINE price indicates that the models are stable dynamically. \\ This suggests that the speed of adjustments is high. \\ The magnitude of the correction of the imbalances, however,suggests for instance that , \textHighlight{61.3\%} of the imbalances in GASOLINE prices are corrected every two weeks
		\end{block}
		
	\end{frame}
	
	% -------------------------------------------------------------------	
	\begin{frame}{VEC Model Coefficients}
		\begin{table}[]
			
			% Gasoline table
			\caption{ \ref{table:gasoil_model} Gasoil Model}
			\label{table:gasoil_model}
			\begin{tabular}{llll}
				\hline
				Parameters      & Coefficient & S.E   & t-satistics \\ \hline
				
				\mc{4}{l}{\textHighlight{Gasoil Model}}                  \\ 
				
				\textSubMath{Gasoline}{t-1} & 0.524  & 0.126  & 4.165*  \\ 
				\textSubMath{Gasoil}{t-1}   & -0.783 & 0.256 & -3.059*  \\ 
				\textSubMath{Kerosene}{t-1} & -0.002 & 0.060 & -0.030   \\ 
				\textSubMath{LPG}{t-1}     & -0.214  & 0.174 & -1.227   \\ 
				Constant        		   & 0.017   & 0.535 & 0.032    \\ 
				ECT             		   & -0.695  & 0.096 & -7.215*  \\ 
				\hline	    
				  
			\end{tabular}
		\end{table}
	
		% Description
		\begin{block}{}
			Concerning GASOIL prices, it indicates 69.5\% of shocks in its prices (imbalance) are corrected every two weeks.
		\end{block}
		
	\end{frame}

	\begin{frame}{VEC Model Coefficients}
		\begin{table}[]
			
			% Gasoline table
			\caption{ \ref{table:kerosene_model} Kerosene Model}
			\label{table:kerosene_model}
			\begin{tabular}{llll}
				\hline
				Parameters      & Coefficient & S.E   & t-satistics \\ \hline
				
				\mc{4}{l}{\textHighlight{Kerosene Model}}                  \\ 
				
				\textSubMath{Gasoline}{t-1} & 0.058 & 0.163 & 0.359 \\
				\textSubMath{Gasoil}{t-1} & -0.002 & -0.085 & -0.256 \\
				\textSubMath{Kerosene}{t-1} & -0.518 & 0.078 & -6.652* \\
				\textSubMath{LPG}{t-1} & 0.059 & 0.225 & 0.263 \\
				Constant & 0.001 & 0.692 & 0.002 \\
				ECT & -0.039 & 0.125 & -0.313 \\
				 
				\hline	    
				
			\end{tabular}
		\end{table}
	
		% Description
		\begin{block}{}
			For the KEROSENE price, 3.9\% of such imbalances are corrected every two weeks
		\end{block}
		
	\end{frame}

	\begin{frame}{VEC Model Coefficients}
		\begin{table}[]
			
			% LPG table
			\caption{ \ref{table:LPG_model} LPG Model}
			\label{table:LPG_model}
			\begin{tabular}{llll}
				\hline
				Parameters      & Coefficient & S.E   & t-satistics \\ \hline
				
				\mc{4}{l}{\textHighlight{LPG Model}}                  \\ 
				
				\textSubMath{Gasoline}{t-1} & 0.054 & 0.106 & 0.505 \\
				\textSubMath{Gasoil}{t-1} & -0.080 & 0.216 & -0.370 \\
				\textSubMath{Kerosene}{t-1} & 0.010 & 0.051 & -0.197 \\
				\textSubMath{LPG}{t-1} & -0.450 & 0.147 & -3.058* \\
				Constant & 0.020 & 0.452 & 0.044 \\
				ECT & -0.036 & 0.081 & -0.437 \\
				
				\hline	    
				
			\end{tabular}
		\end{table}
	
		% Description
		\begin{block}{}
			In the case of LPG price, only 3.6\% of such imbalances are corrected.
		\end{block}
		
	\end{frame}

	\section{Short Run Relationship}
	\begin{frame}{Short Run Relationship}
			
		\begin{block}{}
			The short run relationships of the models are explained by the VEC model coefficients as presented in Table \ref{table:gasoline_model}, \ref{table:gasoil_model}, \ref{table:kerosene_model}, \ref{table:LPG_model}
		\end{block}
		
		\begin{block}{Gasoline}
			Looking at the coefficients, it is observed (Table \ref{table:gasoline_model}) that in the short-run, GASOLINE price \textHighlight{[3.65*]} is significant. This is an indication that GASOLINE price exhibits an increment of \textHighlight{69.1\%} by itself and \textHighlight{2.21\%} reduction by GASOIL price whiles the others are not significant.
		\end{block}
		
	\end{frame}

	\begin{frame}{Short Run Relationship}
		
		\begin{block}{Gasoil}
			Also, it is observed that GASOIL price \textHighlight{[4.17*]} is significant by Gasoline. This is an indication that GASOIL price exhibits an increment of \textHighlight{52.4\%} by GASOLINE price with a \textHighlight{78.3\%} reduction by itself. 
		\end{block}
	
		\begin{block}{}
			The other products also exhibit both increment and reduction by themselves and/or other products. This is because the coefficients of these products are significant. The short-run results also indicate that the variables influence each other.
		\end{block}
		\begin{block}{}
			 Considering GASOLINE price as the dependent variable, it appears the ex- pump prices of the other products influence it. The consequence of this result is that increase ex-pump prices of one or more products are likely to influence others
		\end{block}
		
	\end{frame}
	
	\begin{frame}{Long And Short Run Relationship}
		\begin{block}{}
			\vspaceTen
			Now, having analyzed both the short and long-run relationships \\ existing among the variables, \\
			the VEC models are estimated, diagnosed, and validated, and
			finally, forecasts are generated.
			\vspaceTen
		\end{block}
	\end{frame}
	
	% --------------------------------------------------------------------
	% Estimation of VEC Model
	% --------------------------------------------------------------------
	\section{Estimation of VEC Model}
	\begin{frame}{Estimation of VEC Model}
		The VEC models are estimated using these equations,
		
		\begin{equation}
			\subTT{w}{t} = \text{c} + \subNT{\phi}{1}\subTT{w}{t-1} + \subNT{\phi}{2}\subTT{x}{t-1} + \subNT{\phi}{3}\subTT{y}{t-1} + \subNT{\phi}{4}\subTT{z}{t-1} + \subNT{\epsilon}{t} 
			\label{eq:wt}
		\end{equation}
	
		\begin{equation}
			\subTT{x}{t} = \text{c} + \subNT{\phi}{1}\subTT{w}{t-1} + \subNT{\phi}{2}\subTT{x}{t-1} + \subNT{\phi}{3}\subTT{y}{t-1} + \subNT{\phi}{4}\subTT{z}{t-1} + \subNT{\epsilon}{t} 
			\label{eq:xt}
		\end{equation}
	
		\begin{equation}
			\subTT{y}{t} = \text{c} + \subNT{\phi}{1}\subTT{w}{t-1} + \subNT{\phi}{2}\subTT{x}{t-1} + \subNT{\phi}{3}\subTT{y}{t-1} + \subNT{\phi}{4}\subTT{z}{t-1} + \subNT{\epsilon}{t} 
			\label{eq:yt}
		\end{equation}
	
		\begin{equation}
			\subTT{z}{t} = \text{c} + \subNT{\phi}{1}\subTT{w}{t-1} + \subNT{\phi}{2}\subTT{x}{t-1} + \subNT{\phi}{3}\subTT{y}{t-1} + \subNT{\phi}{4}\subTT{z}{t-1} + \subNT{\epsilon}{t} 
			\label{eq:zt}
		\end{equation}
		
		\begin{block}{}
			The results of VAR are reported by the 4 equations below. The VEC models are computed with one lag. The models relating the products to their lags and that of others may best be described as;
		\end{block}
	\end{frame}

	% --------------------------------------------------------------------
	\begin{frame}
		\begin{equation}
			\begin{bmatrix}
				\subTT{w}{t} \\
				\subTT{x}{t} \\
				\subTT{y}{t} \\
				\subTT{z}{t}
			\end{bmatrix}
			
		\end{equation}
	
	\end{frame}
	
	
	% --------------------------------------------------------------------
	
	
	% --------------------------------------------------------------------
	
	% --------------------------------------------------------------------
	
	% --------------------------------------------------------------------
	% Model Validation
	% --------------------------------------------------------------------
	
	% --------------------------------------------------------------------
	% Forecast of Ex-Pump Prices of Products
	% --------------------------------------------------------------------
	
	% --------------------------------------------------------------------
	% Granger Gausality Test
	% --------------------------------------------------------------------
	
	% --------------------------------------------------------------------
	% Impurse Response Functions ( IRFs )
	% --------------------------------------------------------------------
	
	% --------------------------------------------------------------------
	% Summary of Results
	% --------------------------------------------------------------------
	
	% --------------------------------------------------------------------
	% Summary of Chapter
	% --------------------------------------------------------------------
	
	
	
\end{document}